% \pdfminorversion=4
\documentclass{beamer}
\input{header.tex}
\usetheme{Rochester}


\addtobeamertemplate{navigation symbols}{}{%
    \usebeamerfont{footline}%
    \usebeamercolor[fg]{footline}%
    \hspace{1em}%
    \insertframenumber/16
}

\title{Machine Learning \\ for Gravitational Wave data analysis}
\author{Jacopo Tissino}
\subtitle{Physics of Data workshop}
\date{Venice, 2022-04-08}

\begin{document}

\section{Main presentation}
\frame{\titlepage}

\begin{frame}
    \frametitle{Virgo interferometer}
    \begin{figure}[ht]
        \centering
        \includegraphics[width=\textwidth]{figures/Virgo}
        \label{fig:Virgo}
    \end{figure}
\end{frame}

\begin{frame}
    \frametitle{Bare interferometer data}
    \begin{figure}[ht]
    \makebox[\textwidth][c]{\includegraphics[width=1.3\textwidth]{figures/bare}}
    \label{fig:bare}
    \end{figure}
\end{frame}

\begin{frame}
    \frametitle{Describing Gaussian noise}
    We can completely characterize Gaussian noise through its \textbf{power} or \textbf{amplitude}
    spectral density:
    %
    \begin{align}
    \text{PSD}(f) = S_n (f) = \lim _{T \to \infty} \frac{\abs{\widetilde{d}(f)}^2}{T} 
    \,,
    \end{align}
    \begin{align}
    \text{ASD} (f) = \sqrt{\text{PSD}(f)}
    \,,
    \end{align}
    %
    and then we can whiten the signal as %
    \begin{align}
    \widetilde{d}_w (f) = \frac{\widetilde{d}(f)}{\sqrt{S_n(f)}}
    \,.
    \end{align}
\end{frame}

\begin{frame}
    \frametitle{Amplitude spectral density}
    \vspace{-.2cm}
    \begin{figure}[ht]
    \includegraphics[width=.93\textwidth]{figures/asd}
    \label{fig:asd}
    \end{figure}
\end{frame}

\begin{frame}
    \frametitle{Whitened, bandpassed data}
    \begin{figure}[ht]
    \makebox[\textwidth][c]{\includegraphics[width=1.3\textwidth]{figures/whitened}}
    \label{fig:whitened}
    \end{figure}
\end{frame}

\begin{frame}
    \frametitle{The signal is small}
    \begin{figure}[ht]    
    \makebox[\textwidth][c]{\includegraphics[width=1.25\textwidth]{figures/true_signal}}
    \label{fig:true_signal}
    \end{figure}
\end{frame}

\begin{frame}
    \frametitle{Q-transform}
    \begin{figure}[ht]
    \centering
    \makebox[\textwidth][c]{\includegraphics[width=1.2\textwidth]{figures/q_transform}}
    \label{fig:q_transform}
    \end{figure}
\end{frame}


\begin{frame}
    \frametitle{Signal parametrization}
    The strain at the detector is modelled as \(s(t) = h_\theta (t) + n(t)\), where:
    \begin{itemize}
        \item the noise \(n(t)\) is taken to be stationary, with zero mean, and Gaussian with power spectral density \(S_n(f)\);
        \item the signal \(h_\theta (t)\) can depend on: 
        \begin{itemize}
            \item intrinsic parameters: total mass \(M = m_1 + m_2 \), mass ratio \(q = m_1 / m_2 \), spins \(\vec{\chi}_{1}\) and \(\vec{\chi}_2\), tidal polarizabilities \(\Lambda_1\) and \(\Lambda_2 \);
            \item extrinsic parameters: luminosity distance \(D_L\), inclination \(\iota \)\dots
        \end{itemize}
    \end{itemize}
\end{frame}

\begin{frame}
    \frametitle{The Wiener distance}
    
    The likelihood used in parameter estimation reads:
    %
    \begin{align}
        \Lambda (s | \theta ) \propto \exp( (h_\theta | s) - \frac{1}{2} (h_\theta | h_\theta ))
        \,,
    \end{align}
    %
    where \((a | b)\) is the Wiener product: 
    %
    \begin{align}
    (a | b) = 4 \Re \int_{0}^{\infty } \frac{\widetilde{a}^{*}(f) \widetilde{b} (f)}{S_n (f)} \dd{f}
    = 4 \Re \int_0^{ \infty } a_w^* (f) b_w(f) \dd{f}
    \,.
    \end{align}
\end{frame}

\begin{frame}
    \frametitle{A posterior distribution: GW170817}
    \begin{figure}[ht]
    \vspace{-.2cm}
    \centering
    \includegraphics[width=.7\textwidth]{figures/corner_posterior}
    \label{fig:corner_posterior}
    \end{figure}
\end{frame}

\begin{frame}
    \frametitle{Theoretical signal models}
    The main strategies for the generation of theoretical waveforms are: 
    \begin{itemize}
        \item numerical relativity;
        \item effective one body;
        \item post-Newtonian.
    \end{itemize}

    Other methods mix and match these: hybrid waveforms, phenomenological models, 
    \textbf{surrogates}.
\end{frame}

\begin{frame}
    \frametitle{\texttt{mlgw\_bns} structure}
    \vspace*{-.3cm}
    \begin{figure}[ht]
    \centering
    \includegraphics[width=.775\textwidth]{figures/flowchart}
    \label{fig:flowchart}
    \end{figure}
\end{frame}

\begin{frame}
    \frametitle{Evaluation time}
    \begin{figure}[ht]
    \centering
    \includegraphics[width=.85\textwidth]{figures/benchmarking_evaluation}
    \label{fig:benchmarking_evaluation}
    \end{figure}
\end{frame}

\begin{frame}
    \frametitle{Fidelity}
    \begin{figure}[ht]
    \centering
    \includegraphics[width=.85\textwidth]{figures/mismatches_by_n_train}
    \end{figure}
\end{frame}

\begin{frame}
    \frametitle{More information}
    \begin{itemize}
        \item Learn about GW data analysis at \url{gw-openscience.org};
        \item documentation for \texttt{mlgw\_bns} is available at \url{mlgw-bns.readthedocs.io};
        \item scripts and source for this presentation are available at \url{github.com/jacopok/pod-workshop}.
    \end{itemize}
\end{frame}

\section{Backup slides}

\begin{frame}
    \frametitle{Technologies}
    \texttt{mlgw\_bns} is implemented as a \texttt{python} package, and it makes use of 
    \begin{itemize}
        \item \texttt{scikit-learn} for the neural network (upgrading to \texttt{pytorch});
        \item \texttt{optuna} for the hyperparameter optimization;
        \item \texttt{pytest} and \texttt{tox} for automated testing;
        \item \texttt{numba} for just-in-time compilation and acceleration.
    \end{itemize}
\end{frame}

\begin{frame}
    \frametitle{Original residuals}
    \begin{figure}[ht]
    \centering
    \includegraphics[width=.85\textwidth]{figures/original_residuals}
    \label{fig:original_residuals}
    \end{figure}
\end{frame}


\begin{frame}
    \frametitle{Reconstruction residuals}
    \begin{figure}[ht]
    \centering
    \includegraphics[width=.85\textwidth]{figures/reconstruction_residuals}
    \label{fig:reconstruction_residuals}
    \end{figure}
\end{frame}


\begin{frame}
    \frametitle{Profiling the evaluation: \(\num{8e3}\) interpolation points}
    \begin{figure}[ht]
    \centering
    \includegraphics[width=\textwidth]{figures/sankey_downsampled}
    \label{fig:sankey_downsampled}
    \end{figure}
\end{frame}

\begin{frame}
    \frametitle{Profiling the evaluation: \(\num{2e6}\) interpolation points}
    \begin{figure}[ht]
    \centering
    \includegraphics[width=\textwidth]{figures/sankey_full}
    \label{fig:sankey_full}
    \end{figure}
\end{frame}

\begin{frame}
    \frametitle{Hyperparameter optimization}
    \begin{figure}[ht]
    \centering
    \includegraphics[width=.95\textwidth]{figures/pareto-front}
    \label{fig:pareto-front-nonspinning}
    \end{figure}
\end{frame}

\begin{frame}
    \frametitle{Power Spectral densities and GW170817}
    \begin{figure}[ht]
    \centering
    \includegraphics[width=.91\textwidth]{figures/characteristic_strains}
    \label{fig:characteristic_strains}
    \end{figure}
\end{frame}

\begin{frame}
    \frametitle{Amplitudes}
    \begin{figure}[ht]
    \centering
    \includegraphics[width=.95\textwidth]{figures/native_amplitudes}
    \label{fig:native_amplitudes}
    \end{figure}
\end{frame}

\begin{frame}
    \frametitle{Phases}
    \begin{figure}[ht]
    \centering
    \includegraphics[width=.95\textwidth]{figures/native_phases}
    \label{fig:native_phases}
    \end{figure}
\end{frame}

\end{document}
